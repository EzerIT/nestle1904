\documentclass[11pt,oneside,a4paper,article]{memoir}
\usepackage{fontspec}
\usepackage[unicode=true,xetex,colorlinks=true,linkcolor=blue,urlcolor=blue,bookmarksnumbered=true,bookmarksdepth=3]{hyperref}
\usepackage{bidi} % Must be last

%%%%%%%%%%%%%%%%%%%%%%%%%%%%%%%%%%%%%%%%%%%%%%%%%%%%%%%
%%%%%%%%%%%%%%%%%%% How to typeset %%%%%%%%%%%%%%%%%%%%
%%%%%%%%%%%%%%%%%%%%%%%%%%%%%%%%%%%%%%%%%%%%%%%%%%%%%%%

%% This file must be run through xelatex

%% The following fonts must be accessible to xelatex on
%% your system:
%%
%% Linux Libertine O
%%
%% On Fedora Linux, you can bring in these fonts like
%% so:
%%
%% sudo dnf install linux-libertine-fonts
%%



%%%%%%%%%%%%%%%%%%%%%%%%%%%%%%%%%%%%%%%%%%%%%%%%%%%%%%%
%%%%%%%%%%%%%%%%%%%% Configuration %%%%%%%%%%%%%%%%%%%%
%%%%%%%%%%%%%%%%%%%%%%%%%%%%%%%%%%%%%%%%%%%%%%%%%%%%%%%

%%% Fonts %%%
\setmainfont[Ligatures=TeX]{Linux Libertine O}

\newfontfamily{\mainnolig}{Linux Libertine O}
\newcommand{\q}{{\mainnolig '}}


%%% Page layout %%%
\settypeblocksize{247mm}{160mm}{*}
\setlrmargins{*}{*}{1}
\setulmargins{*}{*}{1}
\checkandfixthelayout

%%% Hyperref (Information in PDF) %%%
\hypersetup{
unicode=true,
pdfauthor={Claus Tøndering},
pdftitle={Generating nestle1904}
}

%%% Lists %%%
\tightlists

%%% Allow extra space between words %%%
\sloppy


%%% Font matter %%%
\title{Generating nestle1904}
\author{Claus Tøndering}
\date{17 May 2023}


\begin{document}
\maketitle


%%%%%%%%%%%%%%%%%%%%%%%%%%%%%%%%%%%%%%%%%%%%%%%%%%%%%%
%%%%%%%%%%%%%%%%%%%% Introduction %%%%%%%%%%%%%%%%%%%%
\chapter{Introduction}

This document gives a brief introduction to how to generate the nestle1904 Emdros database for Bible
Online Learner on a Linux computer.

%%%%%%%%%%%%%%%%%%%%%%%%%%%%%%%%%%%%%%%%%%%%%%%%%%%%%%%
%%%%%%%%%%%%%%%%%%%% Prerequisites %%%%%%%%%%%%%%%%%%%%
\chapter{Prerequisites}

In order to execute this process, you need to \emph{recursively} clone the GitHub repository
\texttt{https://github.com/EzerIT/nestle1904.git}:

\vspace{1ex}

\quad\texttt{git clone --recursive https://github.com/EzerIT/nestle1904}

\vspace{1ex}

In the following description I assume that you have cloned the repository into a folder called
\texttt{nestle1904}.




%%%%%%%%%%%%%%%%%%%%%%%%%%%%%%%%%%%%%%%%%%%%%%%%%%%%%%%%%%%%%%%
%%%%%%%%%%%%%%%%%%%% Generating nestle1904 %%%%%%%%%%%%%%%%%%%%
\chapter{Generating nestle1904}

Executing the command ``make'' in the \texttt{nestle1904/src} folder should generate the nestle1904
Emdros database for Bible OL. This takes about two minutes on my computer.

The steps of the process are detailed in the following sections.


\section{Compiling \emph{nestle2mql}}

The C++ source code for \emph{nestle2mql} is compiled. A C++ compiler that supports the
``-std=c++20'' flag is required.


\section{Executing \emph{nestle2mql}}

\noindent \textbf{Input:} \texttt{../nestle1904-1.2/nestle1904.csv}

\noindent \textbf{Output:} \texttt{nestle.mql}

\vspace{1ex}

\noindent
The file \texttt{../nestle1904-1.2/nestle1904.csv} contains the text to the entire New Testament in
tab-separated lines. The \emph{nestle2mql} uses this to generate MQL code for the \emph{word}
objects of the database.


\section{Compiling \emph{add\_sentences/find\_sentences} and \emph{add\_sentences/maketext}}

The C++ source code for \emph{find\_sentences} and \emph{maketext} in the \emph{add\_sentences}
folder is compiled.


\section{Executing \emph{add\_sentences/maketext}}\label{sec:maketext}

\noindent \textbf{Input:} \texttt{../greek-new-testament/syntax-trees/nestle1904/xml}

\noindent \textbf{Output:} \texttt{add\_sentences/xmlWithNode.txt}

\vspace{1ex}

\noindent
The \emph{maketext} program reads XML files containing the sentence structure of the Greek NT and
generates a file containing \emph{nodeId:word} pairs from those files.

During the process tonos accent marks are replaced by oxia accent marks in the text. The reason for
this is historical. Earlier versions of the XML files used oxia accent marks, and Bible OL still
uses these marks.

For a discussion of the difference between tonos and oxia and a history of their use, see the
section ``A Note on Greek Accents in Unicode'' in the chapter ``Emdros Databases in Bible OL'' in
the Bible OL technical documentation.


\section{Executing \emph{add\_sentences/find\_sentences}}


\noindent \textbf{Inputs:}
\begin{itemize}
\item \texttt{../greek-new-testament/syntax-trees/nestle1904/xml}
\item \texttt{add\_sentences/xmlWithNode.txt}
\end{itemize}

\noindent \textbf{Output:} \texttt{add\_sentences/add\_sentences.mql}

\vspace{1ex}

\noindent
The \emph{find\_sentences} program reads XML files containing the sentence structure of the Greek NT and
generates an MQL file for the \emph{sentence, clause1,} and \emph{clause2} objects of the database.


\section{Applying the MQL Files}

\noindent \textbf{Inputs:}
\begin{itemize}
\item \texttt{nestle1904.mql}
\item \texttt{add\_sentences/add\_sentences.mql}
\end{itemize}

\noindent \textbf{Output:} The \texttt{nestle1904} Emdros database

\vspace{1ex}

\noindent
The ``mql'' command is executed and creates the nestle1904 Emdros database from the MQL files
generated in the previous steps.


\section{Compiling \emph{o2t} and \emph{t2o}}

The C++ source code for the programs \emph{o2t} and \emph{t2o} is compiled.

These two utility programs perform oxia-to-tonos and tonos-to-oxia conversion on their input files.
They are not used in the generation of the nestle1904 database, but they may be useful in other
contexts. (See Section \ref{sec:maketext}.)


\section{Compiling \emph{hintsdb}}

The C++ source code for \emph{hintsdb} is compiled.


\section{Executing \emph{hintsdb}}

\noindent \textbf{Inputs:}
\begin{itemize}
\item The nestle1904 database
\item \texttt{GREEK\_BibleOL\_nominal-ambiguity-project\_v1.21.csv}
\item \texttt{AmbigiousVerbalForms20221021\_BibleOL-export.csv}
\end{itemize}

\noindent \textbf{Output:} The \texttt{nestle1904\_hints.db} database.

\vspace{1ex}

\noindent
The \emph{hintsdb} program is executed. It creates the so-called ``hints database''. For information
about this database, see the chapter \emph{Hints} in the Bible OL technical documentation.

The \texttt{.csv} input files are taken from the corresponding \texttt{.xlsx} files created by
Oliver Glanz. They contain information about various possible interpretations of a particular verbal
or nominal form.



%%%%%%%%%%%%%%%%%%%%%%%%%%%%%%%%%%%%%%%%%%%%%%%%%%%%%%%%%%%%
%%%%%%%%%%%%%%%%%%%% Generating Lexicons %%%%%%%%%%%%%%%%%%%%
\chapter{Generating Lexicons}

There are currently no programs available for generating Greek lexicons.

\end{document}

% Local Variables:
% mode: latex
% ispell-dictionary: "british-ize"
% ispell-extra-args: ("--home-dir=/home/claus/Projects/BibleOL/techdoc")
% eval: (auto-fill-mode 1)
% End:
